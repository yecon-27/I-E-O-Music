\documentclass[manuscript,review, anonymous]{acmart}
\setcopyright{none}
\settopmatter{printacmref=false}
\renewcommand\footnotetextcopyrightpermission[1]{}
\usepackage{booktabs}
\usepackage{tabularx,array}
\usepackage{placeins}

% Reduce spacing around floats
\setlength{\textfloatsep}{4pt plus 1pt minus 1pt}
\setlength{\floatsep}{4pt plus 1pt minus 1pt}
\setlength{\intextsep}{4pt plus 1pt minus 1pt}
\setlength{\abovecaptionskip}{2pt}
\setlength{\belowcaptionskip}{0pt}

% Reduce paragraph spacing
\setlength{\parskip}{0pt plus 1pt}
\setlength{\parsep}{0pt}
\setlength{\topsep}{0pt}
\setlength{\partopsep}{0pt}

\begin{document}

\title{Input–Envelope–Output: Auditable Generative Music Rewards in Sensory-Sensitive Contexts}

\begin{abstract}
Generative feedback in sensory-sensitive contexts poses a core design challenge for system builders. Sensory tolerances vary widely across individuals, requiring a careful balance between engagement and safety. This challenge is exemplified in autism spectrum disorder (ASD), where auditory sensitivities are prevalent yet highly heterogeneous. Existing interactive music systems often conceal safety constraints within direct input–output (I-O) mappings, prioritizing novelty and engagement at the expense of predictable and auditable outcomes. In contrast, we propose a constraint-first Input–Envelope–Output (I-E-O) framework that supports system builders in designing predictable and auditable generative feedback while preserving meaningful action–output causality to sustain engagement. The I-E-O paradigm introduces a low-risk envelope layer between user input and audio output to define safe parameter limits, enforce them deterministically, and log all interventions for audit. Under this approach, we derive four verifiable design principles and instantiate them in a web-based prototype, MusiBubbles. Key contributions include the constraint-first I-E-O architecture, the MusiBubbles prototype demonstrating these design principles, and a reproducibility package. Together, these contributions provide practical infrastructure for system builders designing generative feedback in ASD and other sensory-sensitive domains.
%Generative feedback in sensory-sensitive contexts must balance engagement with safety, given widely varying sensory tolerances among individuals. This challenge is exemplified in autism spectrum disorder (ASD), where auditory sensitivities are prevalent but highly heterogeneous. Existing interactive music systems often conceal safety constraints, maximizing novelty at the expense of predictable outcomes. In contrast, we propose a novel Input-Envelope-Output (I-E-O) framework that prioritizes predictable, auditable feedback within explicit bounds. The I-E-O paradigm introduces a low-risk envelope layer between user input and audio output to define safe parameter limits, enforce them deterministically, and log interventions for audit. We derive four verifiable design principles under this constraint-first approach and instantiate them in a tablet-based prototype, MusiBubbles. Key contributions include the constraint-first I-E-O architecture, the MusiBubbles prototype system demonstrating these four design principles, and a reproducibility package. These contributions provide a foundation for engaging yet safe generative audio feedback in ASD and other sensory-sensitive domains.
\end{abstract}

%Designing generative feedback for sensory-sensitive contexts presents a challenge for system builders: how to balance engagement with safety when sensory tolerances vary widely across individuals? Autism spectrum disorder (ASD) exemplifies this challenge, where auditory sensitivities such as reduced sound tolerance, hyperacusis, and aversion to unpredictable sounds are prevalent yet highly heterogeneous. Many existing interactive music systems implement direct Input–Output mappings, embedding constraints implicitly rather than exposing auditable boundaries. We argue that generative rewards should not default to novelty maximization but instead prioritize predictable, auditable causality within bounded constraints—an Input–Envelope–Output paradigm. This paper presents a constraint-first framework with an explicit low-risk envelope that declares bounds on overload-prone parameters, enforces them via deterministic clamping, and logs all interventions for audit. We formulate four design rationales as verifiable system requirements and instantiate the framework in MusiBubbles, a tablet-based prototype. Key contributions include a reusable constraint-first architecture, four transferable design rationales, and a reproducibility package with engineering evidence. This work provides infrastructure for system builders and sets the stage for future expert review and deployment studies.



\begin{CCSXML}
<ccs2012>
 <concept>
  <concept_id>10010405.10010469.10010475</concept_id>
  <concept_desc>Applied computing~Sound and music computing</concept_desc>
  <concept_significance>500</concept_significance>
 </concept>
 <concept>
  <concept_id>10003120.10011738</concept_id>
  <concept_desc>Human-centered computing~Accessibility</concept_desc>
  <concept_significance>500</concept_significance>
 </concept>
 <concept>
  <concept_id>10003120.10003123</concept_id>
  <concept_desc>Human-centered computing~Interaction design</concept_desc>
  <concept_significance>300</concept_significance>
 </concept>
</ccs2012>
\end{CCSXML}

\ccsdesc[500]{Applied computing~Sound and music computing}
\ccsdesc[500]{Human-centered computing~Accessibility}
\ccsdesc[300]{Human-centered computing~Interaction design}

\keywords{Generative music, sensory-safe design, autism, motor training, constraint-first design}

\maketitle

\section{Introduction}

% 段1: System builder 痛点 + ASD 作为 exemplar
System builders creating interactive applications for sensory-sensitive populations face a fundamental design challenge: generative feedback can sustain engagement, but unconstrained generation risks sensory overload. This tension is particularly acute for individuals with autism spectrum disorder (ASD), where sensory processing differences are reported in over 90\% of the population~\cite{Tomchek2007Sensory,Leekam2007Sensory,BenSasson2009SensoryMeta,Chang2014Sensory}. Auditory sensitivities such as reduced sound tolerance, hyperacusis, and aversion to unpredictable sounds are particularly prevalent~\cite{Robertson2017Sensory}. Critically, these sensitivities are highly heterogeneous, making it difficult to design audio feedback that works for everyone.

% 段2: Motor training 重要性 + audio generative 价值 + 矛盾
Motor training is a cornerstone of daily intervention for children with ASD, supporting the development of coordination, motor planning, and functional skills~\cite{Sigrist2013Augmented,Perochon2023TabletGame}. Audio feedback plays a critical role in sustaining engagement during repetitive exercises~\cite{Sigrist2013Augmented,Danso2025PIMS}, and generative music offers particular promise by producing varied, interaction-shaped outputs that maintain novelty while preserving causal relationships between actions and outcomes~\cite{PlutPasquier2020GenMusicGames,Vlist2011moBeat,Bergstrom2014MusicBiofeedback}. However, designing appropriate audio rewards presents a fundamental tension between constrained and unconstrained feedback. Constrained rewards, such as fixed and preset music, are predictable and safe, but they risk habituation because users lose interest when the same music plays regardless of their actions, which weakens the sense of agency~\cite{LopezDuarte2024PAMG,KojimaNakata2023RepetitiveListeningBoredom,Tabatabaie2014BoredomEEG}. Unconstrained generative rewards can maintain engagement~\cite{Danso2025PIMS,PlutPasquier2019MusicMatters}, but unconstrained generation introduces unpredictability that may trigger sensory overload in sensitive individuals~\cite{Robertson2017Sensory,Williams2021HyperacusisMeta}. Overall, most existing interactive music systems implement direct Input–Output (I-O) mappings, embedding safety-relevant constraints implicitly within musical rules rather than exposing them as auditable, configurable boundaries [1, 30]. This makes it difficult for system builders to tune the balance between habituation-prone constrained rewards and overload-prone unconstrained generation, leaving the effective Input–Output behavior opaque and potentially risky in sensory-sensitive contexts.

%Motor training is a cornerstone of daily intervention for children with ASD, supporting the development of coordination, motor planning, and functional skills~\cite{Sigrist2013Augmented,Perochon2023TabletGame}. Audio feedback plays a critical role in sustaining engagement during repetitive exercises~\cite{Sigrist2013Augmented,Danso2025PIMS}, and generative music offers particular promise by producing varied, interaction-shaped outputs that maintain novelty while preserving causal relationships between actions and outcomes~\cite{PlutPasquier2020GenMusicGames,Vlist2011moBeat,Bergstrom2014MusicBiofeedback}. However, designing appropriate audio rewards presents a fundamental tension. Fixed, preset rewards are predictable and safe but risk habituation: users lose interest when the same music plays regardless of their actions, weakening the sense of agency~\cite{LopezDuarte2024PAMG,KojimaNakata2023RepetitiveListeningBoredom,Tabatabaie2014BoredomEEG}. Generative rewards can maintain engagement~\cite{Danso2025PIMS,PlutPasquier2019MusicMatters}, but unconstrained generation introduces unpredictability that may trigger sensory overload in sensitive individuals~\cite{Robertson2017Sensory,Williams2021HyperacusisMeta}. Most existing interactive music systems implement direct Input–Output mappings, embedding any safety-relevant constraints implicitly within musical rules rather than exposing them as auditable, configurable boundaries~\cite{NIST_AIRMF_100_1_2023,Raji2020AuditFramework}.

% 段3: 我们的方法 + 贡献
To address this gap, this paper presents a constraint-first generative reward framework that interposes an explicit low-risk envelope between user input and audio output. We term this the Input–Envelope–Output (I-E-O) paradigm, in contrast to conventional I-O mappings. The envelope declares conservative bounds on engine parameters (tempo, gain, accent ratio), enforces them via deterministic clamping, and logs all interventions for post-hoc audit. We formulate four design rationales grounding this approach as verifiable system requirements, and instantiate the framework in MusiBubbles, a web-based prototype for post-task music rewards in motor training.

We contribute: (1) four design rationales as transferable principles linking claims to system mechanisms and auditable evidence; (2) a reusable framework architecture with explicit schemas and envelope enforcement; (3) MusiBubbles, a web-based reference implementation; and (4) a reproducibility package with synthetic traces, audit logs, and open-source artifacts.\footnote{Anonymized repository: \url{https://anonymous.4open.science/r/<REDACTED>/}} Our primary target users are system builders (HCI/AI researchers, engineers, designers) creating generative reward feedback for sensory-sensitive contexts. Domain experts (music therapists, ASD practitioners, sensory-processing researchers) serve as gatekeepers who can inform envelope refinement in future ethics-approved studies. Children with ASD and caregivers are ultimate beneficiaries, but clinical effectiveness is out of scope.

\section{Related Work}

\begin{figure*}[!t]
  \centering
  \includegraphics[width=1.0\textwidth]{Flow.pdf}
  \caption{Paradigm shift from Input$\rightarrow$Output to Input$\rightarrow$Envelope$\rightarrow$Output.
  (A) Theory Framework: design space leading to DR1–DR3.
  (B) Abstraction Layer: baseline direct mapping (top) vs our constraint-first pipeline (bottom) with explicit low-risk envelope, compliance gate, and auditable session report.}
  \label{fig:pipeline}
\end{figure*}



\paragraph{Generative music feedback in motor training}
Audio feedback is widely used in motor learning to provide real-time performance information and sustain engagement during repetitive exercises~\cite{Sigrist2013Augmented}. However, fixed audio rewards risk habituation: when the same music plays regardless of user actions, engagement declines and the sense of agency weakens~\cite{BeaudouinLafon2021Generative}. Generative music offers a potential solution by producing varied, interaction-shaped outputs that maintain novelty while preserving causal relationships between actions and outcomes. Tablet-based motor training systems for children with ASD have demonstrated the value of engaging feedback in sustaining participation~\cite{Perochon2023TabletGame}, motivating our focus on generative rewards that can adapt to behavioral patterns.

\paragraph{Sensory-oriented design for ASD}
Several interactive music systems target users with ASD, including Chrome Music Lab~\cite{ChromeMusicLab}, collaborative digital musical interfaces~\cite{Ivanyi2024CADMI}, and uCue for structured musical exploration~\cite{karwankar2025ucue_idc}. Recent work has articulated sensory-oriented design principles for musical interfaces in ASD contexts~\cite{arora2025sensorynime}. However, these systems typically embed safety-relevant constraints implicitly within musical rules rather than exposing them as auditable boundaries. Research on musical preferences in autistic children confirms substantial heterogeneity: Santos et al.~\cite{Santos2024SMC} found that while consonance was consistently preferred, preferences for tempo, pitch, density, and timbre varied across individuals. This motivates explicit, configurable constraints rather than fixed ``safe'' defaults.

\paragraph{Constraint-first and auditable system design}
Our approach draws on established practices from auditable systems and responsible AI. Internal algorithmic auditing frameworks emphasize documentation chains and traceable evidence~\cite{Raji2020AuditFramework}, while risk management frameworks encourage explicit declaration, monitoring, and governance of risk dimensions~\cite{NIST_AIRMF_100_1_2023}. Documentation artifacts such as Model Cards~\cite{Mitchell2019ModelCards} and Datasheets~\cite{Gebru2021Datasheets} provide templates for structured transparency. From systems design, we adopt Design by Contract~\cite{Meyer1992DesignByContract} and the principle of complete mediation~\cite{SaltzerSchroeder1975Protection}, ensuring every parameter passes through the envelope enforcer. Our low-risk envelope operationalizes these principles: bounds are declared upfront, enforced via deterministic clamping, and all interventions logged in session reports for post-hoc audit.

Across these strands of prior work, the I-O mapping emerges as a key locus for balancing engagement and safety in interactive music systems. Yet safety-relevant constraints are often embedded implicitly within musical rules, leaving the effective I-O behavior difficult to inspect, configure, and audit, which motivates our I-E-O framing.


\section{Design Rationales}

We frame these rationales as verifiable system requirements, not clinical safety claims. All bounds represent conservative engineering priors informed by literature; validation with human subjects is planned for future work. Table~\ref{tab:dr_evidence} maps each rationale to its mechanism hook and engineering evidence.

\textbf{DR1: Predictability First.}
In sensory-sensitive contexts, the primary risk of generative audio is not ``poor quality'' but unexpected outputs with excessive variation~\cite{Robertson2017Sensory, Tomchek2007Sensory}. Even deterministic generators can amplify input variability into large swings in stimulation features, making overload risk difficult to anticipate~\cite{Wigham2015IU_Sensory,Boulter2014IU_ASD}. We define predictability as a deployable system property: output variation should be declaratively constrained, with identical inputs producing stable, reproducible results~\cite{Amershi2019Guidelines,NIST_AIRMF_100_1_2023}.

\textbf{DR2: Pattern-Level Mapping.}
Fixed rewards weaken action--outcome expressiveness and risk habituation~\cite{BeaudouinLafon2021Generative,Haggard2017SenseOfAgency}, while 1:1 sonification may amplify behavioral noise into excessive output density~\cite{DubusBresin2013SonificationReview}. We target coarse pattern-level association: rewards reflect aggregate behavioral patterns rather than translating each action into an acoustic event, preserving interaction causality while leaving headroom for envelope constraints.

\textbf{DR3: Low-Risk Envelope.}
Unconstrained generation may produce high-stimulation features (loudness spikes, dynamic jumps, dissonant intervals) increasing sensory overload risk~\cite{Robertson2017Sensory, Takahashi2014Startle}. We operationalize risk mitigation via computable proxy metrics with declarative bounds~\cite{ITU_BS1770_5_2023,EBU_R128_2023}, enforced via deterministic clamping with all interventions logged~\cite{Schneider2000EnforceablePolicies,NIST_SP800_92_2006}.

\textbf{DR4: Auditable, Configurable, Reproducible.}
Sensory sensitivities vary widely; fixed thresholds risk over-generalization~\cite{Gajos2005Preference,Santos2024SMC}. We design configurability as supervised, bounded, and auditable~\cite{Meyer1992DesignByContract,NIST_AIRMF_100_1_2023}: default mode uses conservative settings; tuning mode allows bounded exploration with automatic logging for replay and audit~\cite{Mitchell2019ModelCards,Raji2020AuditFramework}.



\begin{table}[!tbp]
\caption{Design rationales mapped to mechanism hooks and engineering evidence.}
\label{tab:dr_evidence}
\renewcommand{\arraystretch}{1.15}
\setlength{\tabcolsep}{5pt}
\small
\begin{tabularx}{\linewidth}{@{}
  >{\raggedright\arraybackslash}p{0.26\linewidth}
  >{\raggedright\arraybackslash}X
  >{\raggedright\arraybackslash}X @{}}
\toprule
\textbf{Design rationale (DR)} & \textbf{Mechanism hook} & \textbf{Engineering evidence} \\
\midrule
DR1: Predictability as deployable property & Explicit envelope; deterministic clamp/repair & 100\% boundedness (all effective values within declared bounds) \\
DR2: Pattern-level mapping & Pattern labeling $\rightarrow$ template family & Onset density distributions distinguishable across pattern types \\
DR3: Low-risk envelope & Bound overload-prone dimensions; log interventions & Baseline vs enforced distribution shift; clamp rate summary \\
DR4: Auditable, configurable & Requested vs effective params; config hash + seed & Tuning monotonicity across Relaxed/Default/Tight configurations \\
\bottomrule
\end{tabularx}
\end{table}

\section{The I-E-O Framework}

The framework operationalizes the four design rationales into a reusable pipeline with two layers:

\textbf{Instance-specific layer.} \texttt{TaskAdapter} converts task-specific events into a standardized \texttt{actionTrace} schema containing timestamped entries with time, lane/channel, intensity/outcome, and optional note fields. Other discrete motor tasks require only a new adapter.

\textbf{Transferable core.} This layer processes \texttt{actionTrace} through four stages: (1) \texttt{FeatureExtraction} and \texttt{PatternLabeling} (DR2) extract behavioral features and assign pattern labels; (2) \texttt{RewardTemplate} (DR2) selects a template family based on pattern label; (3) \texttt{LowRiskEnvelope} (DR3) declares conservative bounds on engine parameters (tempo, gain, accent ratio); (4) \texttt{EnvelopeEnforcer} (DR3/DR4) implements deterministic clamping with audit logging. Generation parameters (\texttt{rewardSpec}) and enforcement logs (\texttt{SessionReport}) enable full reproducibility.

\textbf{Baseline (I-O) for controlled comparison.}
To make the paradigm contrast testable, we include an explicit baseline that maps \texttt{template} directly to audio (no envelope enforcement, no audit log). This baseline approximates common action-to-output reward implementations and is used only to isolate envelope effects under identical \texttt{actionTrace} and fixed random seeds.

We instantiate this pipeline in MusiBubbles and evaluate it using non-human evidence artifacts aligned with DR1–DR4.

\section{Prototype and Engineering Evaluation}

\subsection{MusiBubbles: Reference Implementation}

MusiBubbles is a reference implementation of the framework, designed to produce standardized \texttt{actionTrace} data, run baseline vs constrained comparisons, and generate \texttt{sessionReport} artifacts for engineering verification and reproducibility. It does not claim therapeutic efficacy or clinical safety.

\textbf{Task and trace schema.} MusiBubbles is a web-based bubble popping game inspired by tablet motor training paradigms~\cite{Axford2018iPad,Perochon2023TabletGame}, featuring five vertical lanes mapping to C Major Pentatonic (C--D--E--G--A), avoiding semitones and tritones to reduce dissonance risk~\cite{Frazee1987Orff}. The system includes an expert panel for real-time monitoring of session parameters and enforcement status. Each 60-second session records tap events into \texttt{actionTrace} entries containing timestamp, lane, intensity/outcome, and optional note fields.

\textbf{Pattern labeling.} Three pattern labels (Sequential, Repetitive, Exploratory) route traces to template families. Labels are assigned via interpretable features: lane diversity, dominant lane ratio, and sequential coverage. When scores fall within 0.05, sequential priority applies as a deterministic tie-breaker. These are pragmatic routing labels, not validated behavioral constructs.

\textbf{Envelope and enforcement.} The low-risk envelope declares bounds on three engine parameters: tempo, gain (dB), and accent ratio (Table~\ref{tab:envelope}). \texttt{EnvelopeEnforcer} applies deterministic clamping and logs all interventions (\texttt{params.requested} vs \texttt{params.effective}) in \texttt{sessionReport}, enabling full replay and audit.

\subsection{Engineering Evaluation}

\textbf{Setup.} We evaluate envelope enforcement using N=660 synthetic \texttt{actionTrace} samples under paired comparison design: each trace runs under baseline (no envelope) and constrained conditions with identical seeds. We test three configurations (Relaxed, Default, and Tight) as shown in Table~\ref{tab:configs} to demonstrate tuning sensitivity.

\textbf{L2 parameter verification (DR1, DR3).} The upper row of Figure~\ref{fig:tuning} shows hockey-stick scatter plots for each L2 parameter. Points on the y=x diagonal (blue) were already within bounds; points deflected from the diagonal (orange) were clamped. Clamp rates scale with constraint tightness: Relaxed achieves only 8.9\% total clamp rate (59/660, all from Gain), confirming the enforcer does not over-intervene when bounds are wide. Default and Tight configurations show progressively higher clamp rates (98.9\% and 99.7\%), with 100\% of \texttt{params.effective} values falling within declared bounds across all conditions.

\textbf{L1 signal diagnostics (DR4).} The lower row of Figure~\ref{fig:tuning} shows Δ distributions (constrained $-$ baseline) for three L1 metrics: onset density, integrated loudness (LUFS), and loudness range (LRA). Distributions shift monotonically with constraint tightness: Tight produces the largest deviations from baseline, Default produces moderate shifts, and Relaxed distributions concentrate near zero. This demonstrates tuning monotonicity: tighter L2 bounds produce proportionally larger L1 effects, enabling practitioners to calibrate constraint strength.

\textbf{Pattern discriminability (DR2).} Onset density distributions remain distinguishable across pattern types even under Tight constraints, confirming that envelope enforcement does not flatten expressivity.

\begin{figure}[t]
    \centering
    \vspace{4.5cm} % Placeholder: ~20 lines height for fig_musibubbles.pdf
    \rule{\linewidth}{0.4pt}
    \caption{MusiBubbles reference implementation and enforcement walkthrough. (A) Expert mode interface showing Safe Range configuration for BPM, Contrast, and Volume. (B) Single-trace enforcement example: input actionTrace (top), baseline vs constrained spectrograms with LUFS contours (middle), and sessionReport excerpt showing requested vs effective parameters (bottom).}
    \label{fig:musibubbles}
\end{figure}

\begin{figure*}[t]
    \centering
    \includegraphics[width=1.0\textwidth]{default.png}
    \caption{Envelope enforcement under Default configuration (N=660 traces). Left three panels (L2 verification): scatter plots show baseline vs constrained values for tempo, gain, and accent ratio; orange points indicate clamped samples (98.9\% total clamp rate). Right three panels (L1 diagnostics): Δ distributions for onset density, LUFS, and LRA demonstrate that L2 bounds produce corresponding L1 signal shifts. Additional configurations (Tight, Relaxed) demonstrating tuning monotonicity are provided in supplementary materials.}
    \label{fig:tuning}
\end{figure*}

\begin{table}[h]
\caption{Two-layer envelope contract: Hard L2 parameters are enforced via clamping and audit-logged; Monitored L1 metrics are diagnostic evidence only.}
\label{tab:envelope}
\small
\renewcommand{\arraystretch}{1.1}
\begin{tabularx}{\columnwidth}{@{}p{0.22\columnwidth}p{0.14\columnwidth}X@{}}
\toprule
\textbf{Parameter} & \textbf{Layer} & \textbf{Evidence} \\
\midrule
\multicolumn{3}{@{}l}{\textit{Hard L2: Engine parameters (enforced + audit logged)}} \\
\midrule
Tempo & L2 (BPM) & Clamp rate; requested$\rightarrow$effective \\
Gain & L2 (dB) & Clamp rate; requested$\rightarrow$effective \\
Accent ratio & L2 (ratio) & Clamp rate; per-dimension distribution \\
\midrule
\multicolumn{3}{@{}l}{\textit{Monitored L1: Signal diagnostics (measured-only, not enforced)}} \\
\midrule
Integrated loudness & L1 (LUFS) & Δ distribution (constrained $-$ baseline) \\
Loudness range & L1 (LU) & Δ distribution; accent ratio effect \\
Onset density & L1 (ev/s) & Δ distribution; tempo effect \\
\bottomrule
\end{tabularx}
\end{table}

\begin{table}[h]
\caption{L2 parameter bounds under three configurations.}
\label{tab:configs}
\small
\renewcommand{\arraystretch}{1.1}
\begin{tabularx}{\columnwidth}{@{}lXXX@{}}
\toprule
\textbf{Parameter} & \textbf{Relaxed} & \textbf{Default} & \textbf{Tight} \\
\midrule
Tempo (BPM) & 60–180 & 120–130 & 124–126 \\
Gain (dB) & $-60$–0 & $-10.5$–$-1.9$ & $-6.9$–$-5.2$ \\
Accent ratio & 0.0–1.0 & 0.0–0.5 & 0.0–0.1 \\
\bottomrule
\end{tabularx}
\end{table}

\section{Limitations and Future Work}

\textbf{Evaluation scope.} Our validation focuses on engineering correctness—demonstrating that parameter bounds are reliably enforced, logged, and reproducible. We have not yet conducted user studies with sensory-sensitive populations, reflecting our contribution's focus on deployable infrastructure. Following ethics approval, we plan expert review sessions with music therapists and accessibility specialists.

\textbf{Domain specificity.} The current implementation targets music-based feedback in movement rehabilitation. Adaptation to other modalities (e.g., haptic feedback, visual effects) would require domain-specific envelope dimensions, though the core enforcement mechanisms remain applicable.

\textbf{Default parameter selection.} Current default bounds derive from published psychoacoustic guidelines. Individual sensory thresholds vary considerably, and the framework supports practitioner-driven customization, though systematic guidelines for bound personalization remain an open research question.

\textbf{Future directions.} We plan to extend the envelope schema to additional dimensions (duration limits, timbre stability), investigate failure modes, and explore adaptation for continuous movement tasks and diverse musical traditions.

\begin{acks}
% Anonymous for review
\end{acks}


\bibliographystyle{ACM-Reference-Format}
\bibliography{references}

\end{document}
